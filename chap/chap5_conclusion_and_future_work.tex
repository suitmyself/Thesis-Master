\chapter{工作总结与展望}
\section{工作总结}
本文重点对计算机图形学领域的弹塑性材料的形变仿真和碎裂仿真进行研究,并提出了一整套基于近场动力学理论的无网格仿真框架。实验结果证明,利用本文所述的弹塑性本构模型、碎裂模型、以及嵌网格策略,能够高逼真度地对物质的弹性形变、塑性形变、脆性碎裂、弹性碎裂、塑性碎裂、和各向异性碎裂进行仿真。回顾本文,主要工作在于以下几个方面:

\begin{itemize}
  \item 在第二章中,本文对于物理仿真的基本原理和碎裂仿真的基本流程进行概述性的解释,并对本构模型、碎裂模型、离散时间积分等重要模块进行介绍。本章的关键还在于详细阐述近场动力学理论的基本特性和本构方程的具体形式,并对从经典连续介质力学到近场动力学的模型演变进行了详细的推导,给出了本文所用动力学模型的基本形式。
  \item 本文第三章主要研究基于近场动力学理论的形变仿真。此章首先对动力学模型进行整理,以更具物理直观的形式重新表述弹性模型。随后,本文给出了全文工作所用塑性模型,整个模型类似于经典理论中的 von Mises 模型,并对其进行改进以获得对于塑性形变的更大控制能力以及提高仿真的整体稳定性。同时,第三章还给出了全文工作所用的离散化框架,我们将粒子置于四面体网格的中心,并通过加权平均的方式来更新网格顶点位置。实践证明,基于上述模型的仿真算法能够对物质的弹塑性行为进行高逼真地仿真。和 FEM 的对比实验也证明,本文工作所用的算法在效果上能够达到和 FEM 共旋线性模型几乎一致的效果。
  \item 本文第四章建立在第三章基础之上,主要研究基于近场动力学理论的碎裂仿真。全章首先描述了全篇工作所用的碎裂模型,本文所用碎裂模型基本上基于 critical stretch,但对其进行了两点重要改进。第一是再次引入权重函数 $\omega_{ij}$,避免离中心粒子越近的邻域粒子反而容易发生断裂。第二是随着碎裂的进行,持续提高碎裂的门槛,以防止产生过多的小碎块。随后,此章还重点介绍了在第三章离散框架基础上,如何进行拓扑更新,以反映碎裂引起的分裂行为。利用上述模型,本文进行了大量的实验工作,并且证明所提框架能够对物质的弹性脆裂、脆性碎裂、塑性碎裂、各向异性碎裂进行准确仿真,同时还能方便处理材料的二次碎裂以及裂纹分支的生长和合并行为。
\end{itemize}

\section{未来展望}
本文对基于近场动力学理论的弹塑性形变和碎裂进行了研究和探讨,并取得了一定的研究成果,但同时也存在诸多不足和亟待完善之处。在此,本文列出这些需要待完善的部分以及其他有意义的研究方向,作为对未来工作的展望。

\begin{itemize}
  \item 碎裂模型及离散化框架的进一步改进。受限于碎裂模型,当前离散化框架将物体仿真粒子置于四面体网格的中心,并沿着单元体边界而进行分裂。虽然处理方便并且取得了不错的效果,但仍需要空间分辨率较高的仿真网格来增加碎裂细节,否则将会产生较为明显的锯齿效果。因此,更实用的离散框架仍然是将仿真粒子直接对应到网格顶点,然后使用基于类似 FEM 中顶点分裂的方式直接对四面体进行剖分。但在近场动力学框架中,这同时需要提出全新的基于粒子而不是使用当前基于 bond 的碎裂模型。如果能够提出合适的碎裂模型,则可以结合 FEM 方法中对于拓扑表示的优点以及近场动力学在物理计算上的优点,更为逼真地仿真碎裂行为。
  \item 近场动力学的隐式时间积分算法。在章节\ref{elasitcity_model}中我们指出,近场动力学的隐式积分算法相对于传统模型要更为复杂,这是由于其理论的非局域性决定的,粒子之间的力计算和整个邻域内的其他粒子相关。如果能够提出对于 state-based 模型的高效隐式时间积分算法,对于进一步挖掘近场动力学的潜力,具有重要意义。
  \item 基于GPU的并行实现。当前的实现采用基于 openMP 的多线程 CPU 并行,但近场动力学模型本身特别适于并行化,因此更高效率的方式是基于 GPU 的并行框架来进行实现,相对于 CPU 多线程实现,GPU 多线程在效率上应该能大幅提升。
  \item FEM 和近场动力学方法的耦合。可以将这两种方法进行耦合,在缺陷之处进行彼此弥补,并同时发挥两者的优势。
\end{itemize}
