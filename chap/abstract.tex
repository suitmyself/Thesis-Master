\begin{cabstract}
在游戏、虚拟现实、电影等工业领域需求日益增长的背景下,基于物理的碎裂仿真一直是计算机图形学中的热点问题,但此问题虽愈发得到重视,却仍极具挑战,至今没有足够成熟的解决方案。碎裂仿真的难点主要在于碎裂模式本身的复杂多样性,以及仿真的稳定性、精确性、和仿真效率。本文基于近场动力学理论,主要针对弹塑性材料的形变和碎裂行为,提出了一种新的无网格动力学仿真框架。

本文所阐述的近场动力学模型能够逼真地对弹塑性材料的形变行为进行表达,通过在实验上多样化材料的属性,然后和传统的 FEM 方法进行充分的对比,结果表明其在效果上能够取得和 FEM 共旋线性(Corotated Linear)模型几乎一致的效果,并且更具稳定性。同时,由于近场动力学本构模型使用的是积分形式的表述,其在仿真的精确性、实现的简单性、以及算法的并行化方面相对于常见的微分形式方法也更具优势。虽然近年 FEM 方法在碎裂仿真领域同样应用广泛,但其作为一种微分方法实际上将在不连续处失去意义。因此,在大形变以及碎裂情况下,往往需要对拓扑网格进行 remeshing 操作来提高网格质量,保证仿真的稳定性,其在实现上较为复杂,计算上较为耗时并且难以并行化。通过结合近场动力学本构模型以及本文所提出的碎裂模型,本文提出的动力学框架能够较好避免这一问题,实验表明本文所提出的方法能够高逼真度地模拟可扩展性材料以及脆性材料的碎裂行为。本文还通过引入各向异性矩阵核来对本构模型进行进一步的拓展,使之可模拟材料的各向异性属性。

在几何表达方面,本工作使用四面体网格来对物体进行离散,并且提出了一种简单的嵌网格策略用于追踪材料刚性运动、形变及碎裂行为。通过此嵌网格策略,也更方便于进行后续物体间的碰撞检测和反应,以及最后的渲染输出工作。

本文是图形学领域第一次使用近场动力学理论来仿真包括弹(脆)性、塑性、碎裂、各向异性等多种现象在内的工作,所提出的方法为基于物理的动力学仿真尤其是碎裂仿真提供了一种新的思路。

\end{cabstract}

\begin{eabstract}
	Test of the English abstract.
\end{eabstract}

