\begin{cabstract}
在虚拟现实、游戏、电影等工业领域需求日益增长的背景下,基于物理的仿真在图形学领域的关注度不断提高。不过由于此问题极具挑战,近年来虽愈发得到重视,但学术界至今仍在探索更为成熟的解决方案。物理仿真的难点主要在于物体本身行为的复杂多样性,以及对仿真的稳定性、精确性、和仿真效率等方面的高要求。本文基于近场动力学理论,主要针对弹塑性材料的形变和碎裂行为,提出了一种新的无网格动力学仿真框架。

本文所阐述的近场动力学本构模型能够逼真地对弹塑性材料的形变行为进行表达,通过在实验上多样化材料的属性,然后和经典的 FEM 方法进行了充分的对比。结果表明其在效果上能够取得和 FEM 共旋线性(Corotated Linear)模型几乎一致的效果,并且更具稳定性。同时,由于近场动力学本构模型使用的是积分形式的表述,其在仿真的精确性、实现的简单性、以及算法的并行化方面相对于传统的微分形式方法也更具优势。虽然近年 FEM 方法在碎裂仿真领域应用广泛,但其作为一种微分方法实际上将在不连续处失去意义。因此,在大形变以及碎裂情况下,往往需要对拓扑网格进行 remeshing 操作来提高网格质量,以保证仿真的稳定性。但其在实现上较为复杂,在计算上较为耗时并且难以并行化。通过结合本文提出的近场动力学本构模型以及碎裂模型,本文所提框架能够较好地避免这一问题,实验表明本文方法能够高逼真度地模拟脆性材料以及可扩展性材料的碎裂行为。本文还通过引入各向异性矩阵核来对本构模型进行了进一步的拓展,使之可模拟材料的各向异性属性。

在几何表达方面,本工作使用四面体网格来对物体进行离散,并且提出了一种简单的嵌网格策略用于追踪材料刚性运动、形变及碎裂行为。通过此嵌网格策略,也更方便于进行后续物体间的碰撞检测和反应,以及最后的渲染输出工作。

本文是图形学领域第一次使用近场动力学理论来仿真包括弹(脆)性、塑性、碎裂、各向异性等多种现象在内的工作,所提出的方法为基于物理的动力学仿真尤其是碎裂仿真提供了一种新的思路。

\end{cabstract}

\begin{eabstract}
Physically-based simulation has been gaining its importance in graphics with the purpose of meeting the growing demand from visual effects industry such as game, virtual reality, and movie.
The main challenge of simulation lies in the diversity behaviors of materials, and high requirements for simulation stability, accuracy, and efficiently.
Albeit a lot of works are devoted to this problem, there still lack elegant solutions due to its extreme complex patterns and behaviors, however.
In this paper, we present a new peridynamics-based meshless framework for the deformation and fracture animation of elastoplastic objects.

Our peridynamics-based elastoplasticity model represents deformation behaviors of materials with high realism.
We validate the model by varying the material properties and performing comparisons with FEM simulations.
The experiments show that we could achieve almost the same visual effects as Corotated-Linear model of FEM while in a more stable manner.
The integral-based nature of peridynamics makes it trivial to model material discontinuities, which outweighs differential-based methods in both accuracy and ease of implementation compared to the traditional differential-based method.
Though recent years has witnessed the widely using of a FEM-based method for fracture modelling, in fact it would break down at discontinuity.
To complement this for simulation stability, FEM usually require a quite daunting remeshing operation that is computational expensive and difficult to be processed in parallel,
to refine the ill-conditioned tetrahedron arising from splitting or large deformation.
By combining the peridynamics-based elastoplastic constitutive model and fracture criterion, the framework could circumvent this drawback largely.
We demonstrate our method could realistically produce ductile fracture as well as brittle fracture.
The constitutive model is further extended by incorporating  a anisotropy model to animate the anisotropic fracture behaviors.

For geometry representation, we employ a tetrahedron mesh for setting peridynamics discretization and propose a simple strategy to explicitly track the deformation and fracture behaviors,
which could also be convenient to collision handling and rendering output in latter work.

Our work is the first application of peridynamics in graphics that could create a wide range of material phenomena including elasticity, plasticity, and fracture.
The complete framework provides an attractive alternative to existing methods for producing modern visual effects.



\end{eabstract}

