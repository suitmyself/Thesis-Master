\documentclass[UTF8]{pkuthss}

\usepackage[backend = biber, style = caspervector, utf8, sorting = ecnty]{biblatex}
\usepackage{amssymb}
\usepackage{amsmath}
\usepackage{graphicx}
\usepackage{deluxetable}
\usepackage{color}

\setlength{\bibitemsep}{3bp}
\renewcommand*{\bibfont}{\zihao{5}\linespread{1.27}\selectfont}

\pkuthssinfo{
	cthesisname = {硕士研究生学位论文}, ethesisname = {Master Thesis},
	ctitle      = {基于近场动力学理论的弹塑性材料碎裂仿真},
    etitle      = {Peridynamics-Based Fracture Animation for Elastoplastic Solids},
	cauthor     = {陈伟},
	eauthor     = {Wei Chen},
	studentid   = {1501214372},
	date        = {二〇一八\ 年\ 六\ 月},
	school      = {信息科学技术学院},
	cmajor      = {计算机软件与理论},
    emajor      = {Computer Software and Theory},
	direction   = {计算机图形学},
	cmentor     = {李胜\ 副教授},
    ementor     = {Sheng Li},
	ckeywords   = {近场动力学,碎裂,弹塑性, 仿真},
    ekeywords   = {peridynamics, fracture, elastoplasticity, animation}
}

\addbibresource{thesis.bib}


\def\pkuthssffaq{%
	\emph{\textcolor{red}{pkuthss 文档模版最常见问题:}}

	\texttt{\string\cite}、\texttt{\string\parencite} %
	和 \texttt{\string\supercite} 三个命令分别产生%
	未格式化的、带方括号的和上标且带方括号的引用标记:%
	\cite{test-en},\parencite{test-zh}、\supercite{test-en, test-zh}。

	若要避免章末空白页,请在调用 pkuthss 文档类时加入 \texttt{openany} 选项。

	如果编译时不出参考文献,
	请参考 \texttt{texdoc pkuthss}“问题及其解决”一章
	“其它可能存在的问题”一节中关于 biber 的说明。
}

\begin{document}

	\frontmatter
	\pagestyle{empty}
	\maketitle
	\cleardoublepage

	% Copyright (c) 2008-2009 solvethis
% Copyright (c) 2010-2017 Casper Ti. Vector
% All rights reserved.
%
% Redistribution and use in source and binary forms, with or without
% modification, are permitted provided that the following conditions are
% met:
%
% * Redistributions of source code must retain the above copyright notice,
%   this list of conditions and the following disclaimer.
% * Redistributions in binary form must reproduce the above copyright
%   notice, this list of conditions and the following disclaimer in the
%   documentation and/or other materials provided with the distribution.
% * Neither the name of Peking University nor the names of its contributors
%   may be used to endorse or promote products derived from this software
%   without specific prior written permission.
%
% THIS SOFTWARE IS PROVIDED BY THE COPYRIGHT HOLDERS AND CONTRIBUTORS "AS
% IS" AND ANY EXPRESS OR IMPLIED WARRANTIES, INCLUDING, BUT NOT LIMITED TO,
% THE IMPLIED WARRANTIES OF MERCHANTABILITY AND FITNESS FOR A PARTICULAR
% PURPOSE ARE DISCLAIMED. IN NO EVENT SHALL THE COPYRIGHT HOLDER OR
% CONTRIBUTORS BE LIABLE FOR ANY DIRECT, INDIRECT, INCIDENTAL, SPECIAL,
% EXEMPLARY, OR CONSEQUENTIAL DAMAGES (INCLUDING, BUT NOT LIMITED TO,
% PROCUREMENT OF SUBSTITUTE GOODS OR SERVICES; LOSS OF USE, DATA, OR
% PROFITS; OR BUSINESS INTERRUPTION) HOWEVER CAUSED AND ON ANY THEORY OF
% LIABILITY, WHETHER IN CONTRACT, STRICT LIABILITY, OR TORT (INCLUDING
% NEGLIGENCE OR OTHERWISE) ARISING IN ANY WAY OUT OF THE USE OF THIS
% SOFTWARE, EVEN IF ADVISED OF THE POSSIBILITY OF SUCH DAMAGE.

% 此处不用 \specialchap,因为学校要求目录不包括其自己及其之前的内容。
\chapter*{版权声明}
% 综合学校的书面要求及 Word 模版来看,版权声明页不需加页眉、页脚。
\thispagestyle{empty}

任何收存和保管本论文各种版本的单位和个人,
未经本论文作者同意,不得将本论文转借他人,
亦不得随意复制、抄录、拍照或以任何方式传播。
否则一旦引起有碍作者著作权之问题,将可能承担法律责任。

% 若需排版二维码,请将二维码图片重命名为“barcode”,
% 转为合适的图片格式,并放在当前目录下,然后去掉下面 2 行的注释。
%\vfill\noindent
%\includegraphics[height = 5em]{barcode}

% vim:ts=4:sw=4



	\cleardoublepage
	\pagestyle{plain}
	\setcounter{page}{0}
	\pagenumbering{Roman}
	% Copyright (c) 2014,2016 Casper Ti. Vector
% Public domain.

\begin{cabstract}
	\pkuthssffaq % 中文测试文字
\end{cabstract}

\begin{eabstract}
	Test of the English abstract.
\end{eabstract}

% vim:ts=4:sw=4


	\tableofcontents

	\mainmatter
	% Copyright (c) 2014,2016 Casper Ti. Vector
% Public domain.

\specialchap{序言}
\pkuthssffaq % 中文测试文字。

% vim:ts=4:sw=4

	% Copyright (c) 2014,2016 Casper Ti. Vector
% Public domain.

\chapter{章节}
\pkuthssffaq % 中文测试文字。

% vim:ts=4:sw=4

	% Copyright (c) 2014,2016 Casper Ti. Vector
% Public domain.

\specialchap{结论}
\pkuthssffaq % 中文测试文字。

% vim:ts=4:sw=4



	\appendix
	\printbibliography[heading = bibintoc]

	\backmatter
	\chapter{致谢}
感谢我的导师李胜副教授。李老师是我从天体物理转向计算机图形学方向的领路人,无论是在学术上还是人生道路上,都给予了我很大帮助。感谢汪国平教授,其工作态度永远值得我们学习。

感谢朱飞师兄,没有朱飞师兄对我的提携和不断督促,以及对本工作的不断完善,本工作将是错漏百出、举步维艰的。感谢何小伟师兄、杨升师兄、徐力有师弟、朱奎鑫师弟、田然师弟在科研讨论时提供的有效思路及帮助。

感谢北京大学,从本科到硕士,度过了我最为美好、最为充实的七年时光。

最后,感谢那些还陪伴在我身边的人以及未来会陪伴在我身边的人,你们的存在是我不断努力、奋勇向前的最大动力与意义。

人生但苦无妨。未来,愿付出所有,换前所未有。


	% Copyright (c) 2008-2009 solvethis
% Copyright (c) 2010-2017 Casper Ti. Vector
% All rights reserved.
%
% Redistribution and use in source and binary forms, with or without
% modification, are permitted provided that the following conditions are
% met:
%
% * Redistributions of source code must retain the above copyright notice,
%   this list of conditions and the following disclaimer.
% * Redistributions in binary form must reproduce the above copyright
%   notice, this list of conditions and the following disclaimer in the
%   documentation and/or other materials provided with the distribution.
% * Neither the name of Peking University nor the names of its contributors
%   may be used to endorse or promote products derived from this software
%   without specific prior written permission.
%
% THIS SOFTWARE IS PROVIDED BY THE COPYRIGHT HOLDERS AND CONTRIBUTORS "AS
% IS" AND ANY EXPRESS OR IMPLIED WARRANTIES, INCLUDING, BUT NOT LIMITED TO,
% THE IMPLIED WARRANTIES OF MERCHANTABILITY AND FITNESS FOR A PARTICULAR
% PURPOSE ARE DISCLAIMED. IN NO EVENT SHALL THE COPYRIGHT HOLDER OR
% CONTRIBUTORS BE LIABLE FOR ANY DIRECT, INDIRECT, INCIDENTAL, SPECIAL,
% EXEMPLARY, OR CONSEQUENTIAL DAMAGES (INCLUDING, BUT NOT LIMITED TO,
% PROCUREMENT OF SUBSTITUTE GOODS OR SERVICES; LOSS OF USE, DATA, OR
% PROFITS; OR BUSINESS INTERRUPTION) HOWEVER CAUSED AND ON ANY THEORY OF
% LIABILITY, WHETHER IN CONTRACT, STRICT LIABILITY, OR TORT (INCLUDING
% NEGLIGENCE OR OTHERWISE) ARISING IN ANY WAY OUT OF THE USE OF THIS
% SOFTWARE, EVEN IF ADVISED OF THE POSSIBILITY OF SUCH DAMAGE.

{
	\ctexset{section = {
		format+ = {\centering}, beforeskip = {40bp}, afterskip = {15bp}
	}}

	% 学校书面要求本页面不要页码,但在给出的 Word 模版中又有页码且编入了目录。
	% 此处以 Word 模版为实际标准进行设定。
	\specialchap{北京大学学位论文原创性声明和使用授权说明}
	\mbox{}\vspace*{-3em}
	\section*{原创性声明}

	本人郑重声明:
	所呈交的学位论文,是本人在导师的指导下,独立进行研究工作所取得的成果。
	除文中已经注明引用的内容外,
	本论文不含任何其他个人或集体已经发表或撰写过的作品或成果。
	对本文的研究做出重要贡献的个人和集体,均已在文中以明确方式标明。
	本声明的法律结果由本人承担。
	\vskip 1em
	\rightline{%
		论文作者签名:\hspace{5em}%
		日期:\hspace{2em}年\hspace{2em}月\hspace{2em}日%
	}

	\section*{%
		学位论文使用授权说明\\[-0.33em]
		\textmd{\zihao{5}(必须装订在提交学校图书馆的印刷本)}%
	}

	本人完全了解北京大学关于收集、保存、使用学位论文的规定,即:
	\begin{itemize}
		\item 按照学校要求提交学位论文的印刷本和电子版本;
		\item 学校有权保存学位论文的印刷本和电子版,
			并提供目录检索与阅览服务,在校园网上提供服务;
		\item 学校可以采用影印、缩印、数字化或其它复制手段保存论文;
		\item 因某种特殊原因需要延迟发布学位论文电子版,
			授权学校在 $\Box$\nobreakspace{}一年 /
			$\Box$\nobreakspace{}两年 /
			$\Box$\nobreakspace{}三年以后在校园网上全文发布。
	\end{itemize}
	\centerline{(保密论文在解密后遵守此规定)}
	\vskip 1em
	\rightline{%
		论文作者签名:\hspace{5em}导师签名:\hspace{5em}%
		日期:\hspace{2em}年\hspace{2em}月\hspace{2em}日%
	}

	% 若需排版二维码,请将二维码图片重命名为“barcode”,
	% 转为合适的图片格式,并放在当前目录下,然后去掉下面 2 行的注释。
	%\vfill\noindent
	%\includegraphics[height = 5em]{barcode}
}

% vim:ts=4:sw=4

\end{document}
